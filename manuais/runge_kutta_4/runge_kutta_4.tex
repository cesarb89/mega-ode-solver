\documentclass[12pt,a4paper]{article}%
\usepackage{amsmath,amsfonts,amstext,amsthm}%
\usepackage{pstricks,pst-plot}%
\usepackage[dvips]{graphicx}%
\usepackage[brazil]{babel}%
\usepackage[utf8]{inputenc}%

\usepackage{enumerate}

\pagestyle{empty}%

\setlength{\topmargin}{-2.5cm} 
\setlength{\oddsidemargin}{-1cm} %%% top margin -3
\setlength{\evensidemargin}{1cm} 
\setlength{\textheight}{27cm}
\setlength{\textwidth}{18cm}

\newcommand{\comando}[1]{{\Large #1} \\ \noindent\rule {17.9cm}{0.05cm}}
\newcommand{\parte}[1]{\vspace{1cm}{\large #1} \\ \noindent\rule {17.9cm}{0.05cm}}


\begin{document}


\comando{runge\_kutta\_4}

Método de Runge-Kutta de 4a ordem

\parte{Syntax}

\texttt{[y,t]=runge\_kutta\_4(f,t0,y0,h,N)}

Input:
\begin{itemize}
\item \texttt{f} is the 
\item \text{t0} is the 
\item \texttt{y0} is the 
\item \texttt{h} is the 
\item \texttt{N} is the 
\end{itemize}

Output:

\begin{itemize}
\item \texttt{y} is the 
\item \texttt{t} is the 
\end{itemize}


\parte{Description}

Descrição do método

\parte{Example}

\begin{verbatim}
clc
clear all
close all

Exemplo 

\end{verbatim}

\end{document}









